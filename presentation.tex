%
%  untitled
%
%  Created by Hidde-Jan Jongsma on 2010-10-22.
%  Copyright (c) 2010 __MyCompanyName__. All rights reserved.
%
\documentclass[]{beamer}

% Use utf-8 encoding for foreign characters
\usepackage[utf8]{inputenc}

\usetheme{Malmoe}


% Setup for fullpage use
% \usepackage{fullpage}

% Uncomment some of the following if you use the features
%
% Running Headers and footers
%\usepackage{fancyhdr}

% Multipart figures
%\usepackage{subfigure}

% More symbols
\usepackage{amsmath}
\usepackage{amssymb}
\usepackage{latexsym}

% Surround parts of graphics with box
\usepackage{boxedminipage}

% Package for including code in the document
\usepackage{listings}

% If you want to generate a toc for each chapter (use with book)
% \usepackage{minitoc}

% This is now the recommended way for checking for PDFLaTeX:
\usepackage{ifpdf}

\newcommand{\e}{\epsilon}
\newcommand{\s}{\sigma}
\newcommand{\be}{\beta}
\newcommand{\g}{\gamma}
\newcommand{\om}{\omega}
\newcommand{\T}{\mathbb{T}}
\newcommand{\KAM}{KAM  }
\newcommand{\rst}{\! \mid}

%\newif\ifpdf
%\ifx\pdfoutput\undefined
%\pdffalse % we are not running PDFLaTeX
%\else
%\pdfoutput=1 % we are running PDFLaTeX
%\pdftrue
%\fi

% \ifpdf
% \usepackage[pdftex]{graphicx}
% \else
% \usepackage{graphicx}
% \fi
\title{\KAM Theory}
\author{ Hidde-Jan Jongsma }
\institute{ RuG }

\date{\today}

\begin{document}

% \ifpdf
% \DeclareGraphicsExtensions{.pdf, .jpg, .tif}
% \else
% \DeclareGraphicsExtensions{.eps, .jpg}
% \fi

\maketitle

\section{Introduction}

\begin{frame}{Outline}
    \small \tableofcontents
\end{frame}

\subsection{What is KAM Theory}

\begin{frame}{What is KAM Theory?}
    \begin{itemize}
        \item Is quasi-periodicity \emph{natural}?
        \item Solar System is multi-/quasi-periodic?
        \item Are quasi-periodic subsystems persistent?
        \item Kolmogorov, Arnold and Moser
    \end{itemize}
\end{frame}

\subsection{Multi-/Quasi-periodic subsystems}

\begin{frame}{Multi- and quasi-periodic subsystem}
    
    \begin{definition}
        A system $(M, T, \Phi)$ has a multi-periodic subsystem
        if $\exists (\mathcal{T}, T, \Phi_\Omega \rst_{\mathcal{T}})$ and a 
       translation system where
        \begin{equation*}
            \Phi_{\Omega}([x],t) = R_{t \Omega}([x]) = [x + t \Omega]
        \end{equation*}
        and $f : \T^n \rightarrow M$, $f(\T^n) = \mathcal{T}$ s.t.
        \begin{equation*}
            f(\Phi_{\Omega}([x],t)) = \Phi(f([x]), t)
        \end{equation*}
    \end{definition}
    
    \pause

    What happens when we perturb $\Phi \to \tilde{\Phi}$; $\mathcal{T} \to \tilde{\mathcal{T}}$?
    
\end{frame}

\section{KAM Theory of circle maps}

\subsection{Periodically driven and coupled oscillators}

\begin{frame}{Periodically driven and coupled oscillators}
    
    Van der Pol oscillator in $C^\infty$-format:
    \begin{equation*}
        y'' + cy' + ay + f(y,y') = \epsilon g(y,y',\omega t; \epsilon),
    \end{equation*}
    \pause
    $f$ such that the free system has a \emph{hyperbolic attractor}, $g$ $2\pi$-periodic.
    
    \medskip
    Coupled oscillator:
    \begin{align*}
        y_1'' + cy_1' + ay_1 + f_1(y_1,y_1') = & \ \e g_1(y_1,y_1',y_2,y_2') \\
	y_2'' + cy_2' + ay_2 + f_2(y_2,y_2') = & \ \e g_2(y_1,y_1',y_2,y_2').
    \end{align*}

    \pause
    
    These systems give rise to vectors field $X_\e$.
    
    \pause
    \begin{center}
        \large
        Are the dynamics of $X_0 \rst_{T_0}$ persistent?
    \end{center}
    
    
\end{frame}

% \begin{frame}{Periodically driven oscillator}
%     $X_0$ has a multi-periodic subsystem $T_0$. 
%     $X_{0} \! \mid_{T_0}$ conjugated to translational system
%     \begin{align*}
%         x_1' & = \om_1 \\
%         x_2' & = \om_2,
%     \end{align*}
%     on $\T^2$ where $x_1 = x, \ \om_1 = \om$.
%     
%     \pause
%     
%     \medskip
%     
%     Since $T_0$ is normally hyperbolic, by the 
%     Normally Hyperbolic Invariant Manifold Theorem,
%     $T_0$ is persistent as invariant manifold:
%     \begin{center}
%         $X_\e$ \emph{also} has invariant manifold $T_\e$, smoothly dependent on $\e$.
%     \end{center}
% \end{frame}
% 
% \begin{frame}
%     \begin{center}
%         \LARGE
%         Are the dynamics of $X_0 \rst_{T_0}$ persistent?
%     \end{center}
% \end{frame}

\subsection{Poincar\'{e} return maps}

% \begin{frame}{Poincar\'{e} return maps}
%     We want to find a smooth conjugation between $X_0$ and $X_\e$.
%     
%     \medskip
%     
%     \textbf{Problem:} Disturbances can turn quasi-periodic orbits in periodic orbits.
%     
%     \pause
%     
%     \medskip
%     
%     \textbf{Solution:} Introduce parameters.
%     
%     \begin{equation*}
%         \mu = (\om_1, \om_2) \pause
%         \rightarrow \be = \om_2 / \om_1.
%     \end{equation*}
%     
% \end{frame}

\begin{frame}{Poincar\'{e} return maps}
    We can bring this problem
    from $\T^2, T=\mathbb{R}$ to $\T^1, T=\mathbb{Z}$ by taking Poincar\'e map
%     can now study $X_{\be, \e}$ on $\T^2$.
%     This corresponds to the family of Poincar\'{e} return maps
    \begin{equation*}
        P_{\be,\e} : x \to x + 2 \pi \be + \e a(x, \be, \e),
    \end{equation*}
    or, equivalently, as a fibre preserving map
    \begin{equation*}
        P_{\e} : (x, \be) \to (x + 2 \pi \be + \e a(x, \be, \e), \be).
    \end{equation*}
    \pause
    Since an conjugation between $P_0$ and $P_\e$ means an
    equivalence between $X_0$ and $X_\e$, we want $\Phi_\e$ s.t.
    \begin{equation*}
        P_\e \circ \Phi_\e = \Phi_\e \circ P_0.
    \end{equation*}
    
\end{frame}

\section{Linear small divisor problem}

\subsection{Finding $\Phi_\e$: Fourier series}

\begin{frame}{Finding $\Phi_\e$: Fourier series}
    We assume $\Phi_\e$ has the form
    \begin{equation*}
        \Phi_\e (x,\be) = (x + \e U(x,\be,\e), \be + \e \sigma(\be, \e)).
    \end{equation*} \pause
    Since $\Phi_\e$ should conjugate $P_0$ to $P_\e$, we obtain
    a non-linear equation
    \begin{align*}
	    U(x + 2 \pi & \be,\be,\e) - U(x, \be, \e) \\
            & = 2 \pi \sigma(\be, \e) + a(x + \e U(x, \be, \e), \be + \e \sigma(\be, \e), \e).
    \end{align*}
    \pause
    Very hard to solve, we first linearize to
    \begin{equation}\label{eq:firstord}
        U_0(x + 2 \pi \be,\be) - U_0(x, \be)
        = 2 \pi \sigma_0(\be) + a_0(x, \be),
    \end{equation}
    which we solve by Fourier series.
\end{frame}

\subsection{Diophantine conditions}

\begin{frame}{Diophantine conditions}
    This yields
    \begin{equation}\label{eq:formalsol}
        \begin{split}
            a_0(x, \be) = & \sum_{k \in \mathbb{Z}} a_{0 k}(\be)e^{ikx},
            \quad \sigma_0(\be) = \frac{1}{2 \pi} a_{00}(\be) \text{ and} \\
            & U_0(x, \be) = \sum_{k \in \mathbb{Z}} \frac{a_{0 k}(\be)}{e^{2 \pi ik\be} - 1} e^{ikx}.
        \end{split}
    \end{equation}
    \pause
    \begin{definition}
        For $\tau > 1, \g > 0$. $\be \in [0, 1]$ is called $(\tau + 1, \g)$-Diophantine, if
        $\forall p,q \in \mathbb{Z}, q > 0$
        \begin{equation*}
            \left| \be - \frac{p}{q} \right| \ge \frac{\g}{q^{\tau + 1}},
        \end{equation*}
    \end{definition}
\end{frame}

\begin{frame}
            $[0,1]_{\tau+1, \g} := \{ \be \mid \be (\tau + 1, \g)$-Diophantine$\}$.
\begin{itemize}
    \item Union of Cantor set and countable set.
    \item $\operatorname{measure}([0,1]-[0,1]_{\tau+1, \g}) \le 2 \g \displaystyle \sum_{q \ge 1} q^{-\tau}$
\end{itemize}
\pause
    \begin{theorem}
        If $\be$ is $(\tau + 1, \g)$-Diophantine, then
        \begin{equation*}
            \left| e^{2 \pi i k \be} - 1 \right| \ge \frac{4 \g}{|k|^\tau}.
        \end{equation*}
    \end{theorem}
\end{frame}

\begin{frame}
    \begin{proof}
        Suppose $\be$ is $(\tau + 1, \g)$-Diophantine, $k > 0$, now
        \begin{equation*}
            |e^{2\pi i k \beta}  - 1| = 2 \sin (\pi k \beta).
        \end{equation*}
        For $0 < x < \frac{\pi}{2}$, $\sin(x) \ge \frac{2 x}{\pi}$, so
        \begin{equation*}
            |e^{2\pi i k \beta}  - 1| = 2 \sin (\pi k \be) \ge 2 k \be.
        \end{equation*}
        Since $\be$ satisfies Diophantine conditions, it follows that
        \begin{align*}
            \be \ge \frac{\gamma}{k^{\tau + 1}} \quad \text{ and } \quad
            |e^{2\pi i k \beta}  - 1| \ge \frac{4 \g}{k^\tau}.
        \end{align*}
    \end{proof}
\end{frame}

\subsection{Linear small divisor problem}

\begin{frame}{1-bite small divisor problem}
    We generalize \eqref{eq:firstord} to
    \begin{equation}\label{eq:lin}
        u(x + 2 \pi \be, \be) - u(x, \be) = f(x,\be),
    \end{equation}
    if we take the averages of both sides, it follows
    that necessarily $f$ is zero on average,
    bounded by $M$ in a complex neighborhood around $\T^1$.
\pause
    \begin{lemma}[Paley-Wiener, scalar case]
        Let $f = f(x,\om)$ real analytic as above, with Fourier series
        \begin{equation*}
            f(x) = \sum_{k \in \mathbb{Z}} f_k(\om)e^{ikx}.
        \end{equation*}
        Then, $\forall k \in \mathbb{Z}$, $\om$ in appropiate neighborhood,
	\begin{equation*}
	    |f_k(\om)| \le M e^{-\kappa |k|}
	\end{equation*}
    \end{lemma}
\end{frame}

\begin{frame}

\begin{proof}
    Now $f_k$ is defined as
    \begin{equation*}
        f_k(\om) = \frac{1}{2\pi} \oint_{\T^1} f(x,\om)e^{-ikx} dx.
    \end{equation*}
    We expand $f$ holomorphically to a small complex strip around $\T^1$ given by
    \begin{equation*}
        z = x - i \kappa \operatorname{sign}(k).
    \end{equation*}
    \pause
    We obtain from the Cauchy Theorem
    \begin{equation*}
        f_k(\om) = \frac{1}{2\pi} \oint f(z,\om)e^{-ikz} dz
        = \frac{1}{2\pi} \oint_{\T^1} f(x,\om)e^{-ikx - \kappa|k|} dx.
    \end{equation*}
    Hence,
	\begin{equation*}
	    |f_k(\om)| \le \frac{e^{- \kappa|k|}}{2\pi} \oint_{\T^1} |f(x,\om)| dx \le M e^{-\kappa |k|}.
	\end{equation*}
\end{proof}
\end{frame}

\subsection{Convergence of the Fourier series}

\begin{frame}{Convergence of the Fourier series}
    Now we take the Fourier series of $u$
    \begin{equation*}
        u(x) = \sum_{k \in \mathbb{Z}} u_k e^{i k x}.
    \end{equation*}
    From \eqref{eq:lin} it follows that
    \begin{equation*}
        u_k = \frac{f_k}{e^{2\pi i k \be} - 1}.
    \end{equation*}
    \pause
    It follows that
    \begin{equation*}
        |u_k| = \frac{|f_k|}{|e^{2\pi i k \be} - 1|} 
            \le \frac{|k|^\tau M}{4 \g} e^{-\kappa |k|}.
    \end{equation*}
\end{frame}

\begin{frame}
    % Now
    % \begin{equation*}
    %     |u_k| = \frac{|f_k|}{|e^{2\pi i k \be} - 1|} 
    %         \le \frac{|k|^\tau M}{4 \g} e^{-\kappa |k|}.
    % \end{equation*}
    Since $|k|^\tau$ polynomial, there exist $M', \kappa' < \kappa$
    s.t.
    \begin{equation*}\label{eq:ukbound}
        |u_k| 
            \le \frac{|k|^\tau M}{4 \g} e^{-\kappa |k|}
        \le
            M' e^{- \kappa' |k|}.
    \end{equation*}
    \pause
    It follows that $\sum_{k \in \mathbb{Z}} u_k e^{ikx}$ converges
    on a complex strip 
    \begin{equation*}
        \T^1 + \kappa' = \{ x \in \mathbb{C}/(2 \pi \mathbb{Z}) \mid |\Im(x)| < \kappa' \}.
    \end{equation*}
    \pause
    On this strip, $u$ is bounded by
    \begin{equation*}
        |u(x,\om)| \le \sum_{k \in \mathbb{Z}} | u_k | 
        \le M' \sum_{k \in \mathbb{Z}}  e^{-\kappa' |k|}
        = M' \frac{e^{\kappa'} + 1}{e^{\kappa'} - 1},
    \end{equation*}
    and, by the Paley-Wiener Lemma, is an analytic function.
\end{frame}

\begin{frame}{Linear small divisor problem, real-analytic case}
    \begin{theorem}
        Considering the 1-bite small divisor problem with real-analytic 
        $f$ with average $0$. $\forall \be \in [0,1]_{\tau +1, \g}$, $\tau > 1$,
        the formal solution \eqref{eq:formalsol} converges to
        a real-analytic solution.
    \end{theorem}
    \begin{proof}
        \begin{equation*}
            \operatorname{avg}(2 \pi \sigma_0(\be) + a_0(x,\be)) = - a_{00}(\be) + a_{00}(\be) = 0.
        \end{equation*}
        The Fourier coefficients of $U_0(x, \be)$ decrease exponentially and so by
        the Paley-Wiener Lemma, $U_0$ is real-analytic.
    \end{proof}
\end{frame}

\section{Newtonian iteration}

\subsection{Back to circle maps}

\begin{frame}{Back to circle maps}
    We have found a real-analytic solution to
    \begin{equation*}
        U_0(x + 2 \pi \be,\be) - U_0(x, \be)
        = 2 \pi \sigma_0(\be) + a_0(x, \be),
    \end{equation*}
    but this does not guarantee that a solution $U(x, \be, \e)$ exists to
    \begin{align*}
	    U(x + 2 \pi & \be,\be,\e) - U(x, \be, \e) \\
            & = 2 \pi \sigma(\be, \e) + a(x + \e U(x, \be, \e), \be + \e \sigma(\be, \e), \e).
    \end{align*}
    \pause
    We can apply a Newtonian iteration process to obtain $U$ and $\Phi_\e$.
    
\end{frame}

\subsection{Newtonian iteration}
\begin{frame}{Newtonian iteration}
    We can obtain $\Phi_\e$ as the limit of a sequence $\{ \Phi_{\e,i} \}_{k\ge0}$,
    starting at $\mathrm{Id}$.

\medskip
\pause
    At each step, we solve a linear equation of the form
\begin{equation*}
    \tilde U (x + 2\pi \be, \be) - \tilde U (x,\be) = 2 \pi \tilde \sigma(\be) + a(x, \be),
\end{equation*}
   were we try to reduce `error' $a$. We have seen how we can do this.
\end{frame}

\begin{frame}
    For the first iteration, we know there exists a solution $\Phi_{\e,0}$ on
    a certain complex domain $D_1$
    \begin{equation*}
        D_1 = \{ (x,\be) \in \mathbb{C}/(2 \pi \mathbb{Z}) \times [0,1]_{\tau +1, \g} \mid |\Im(x)| < \tilde \kappa \},
    \end{equation*}
     which maps $D_1$ back to $D_0$ and from which we obtain $P_{\e,0}$ on $D_1$.
     
\medskip
\pause
In the next step, we start at $P_{\e,1}$ and obtain $\Psi_1$, now
\begin{equation*}
    \Psi_1 : D_2 \rightarrow D_1 \text{ and } \Phi_{\e,1} = \Phi_{\e,0} \circ \Psi_{1}.
\end{equation*}

\pause

The sequence of real-analytic $\Phi_{\e, i}$'s converges to $\Phi_\e$.
\end{frame}




% \section{Torus flows}
% 
% \subsection{Torus flows}
% 
% \begin{frame}{Torus flows}
%     These are a generalization of circle maps.
% \end{frame}
% 
% \section{Putting it all together}
% 
% \subsection{Convergence of Fourier series}
% 
% \begin{frame}{Convergence of Fourier series}
%     \begin{itemize}
%         \item Paley-Wiener Lemma
%         \item Diophantine vectors
%         \item $(\om_1, \om_2)$ Diophantine vector $\rightarrow \be = \om_2/\om_1$ Diophantine number
%     \end{itemize}
% \end{frame}
% 
\subsection{Circle map theorem}

\begin{frame}{Circle map theorem}
    \begin{theorem}
        For $\tau > 1, \g$ suff.\ small, $\e a$ suff.\ small in $C^\infty$ sense,
        there exists a $C^\infty$ transformation
        \begin{equation*}
            \Phi_\e : \T^1 \times [0,1] \rightarrow \T^1 \times [0,1]
        \end{equation*}
        that conjugates $P_0 \! \mid_{[0,1]_{\tau + 1, \g}}$ to
        a quasi-periodic subsystem of $P_\e$.
    \end{theorem}

\medskip
\pause
Quasi-periodic behavior is persistant for Circle maps (under right conditions).
We have \emph{quasi-periodic stability}.
\end{frame}

\begin{frame}{Extention to $\T^n$}
\pause
\begin{itemize}
    \item Diophantine vectors
    \begin{equation*}
         |\langle \omega, k \rangle | \ge \gamma|k|^{-\tau} \quad [k \in \mathbb{Z}^n]
    \end{equation*} 
\vspace{-.5cm}
\pause
    \item Torus flows 
\pause
    \item Newtonian iteration on vector fields
\end{itemize}

\end{frame}


\bibliographystyle{plain}
\bibliography{}
\end{document}
