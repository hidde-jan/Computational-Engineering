\documentclass[a4paper]{article}

\author{Hidde-Jan Jongsma}
\title{Finite Element Method}
\date{\today}

% Extra symbols and enviroments
\usepackage{amsmath}
\usepackage{amsfonts}
\usepackage{mathtools}

% Use full page
\usepackage{fullpage}

% Create graphics in LaTex
\usepackage{tikz}

% Extra commands
\newcommand{\Reals}{\mathbb{R}}

\newcommand{\dux}{\frac{\partial u}{\partial x}}
\newcommand{\duxx}{\frac{\partial^2 u}{\partial x^2}}
\newcommand{\duxxx}{\frac{\partial^3 u}{\partial x^3}}

\newcommand{\duy}{\frac{\partial u}{\partial y}}
\newcommand{\duyy}{\frac{\partial^2 u}{\partial y^2}}
\newcommand{\duyyy}{\frac{\partial^3 u}{\partial y^3}}
\newcommand{\lapu}{\nabla^2 u}

\providecommand{\abs}[1]{\lvert#1\rvert}
\providecommand{\norm}[1]{\lVert#1\rVert}

\newcommand{\LO}{\ensuremath{L_2(\Omega)}}
\newcommand{\HO}{\ensuremath{H^1(\Omega)}}
\newcommand{\HOzero}{\ensuremath{H^1_{(0}(\Omega)}}


\begin{document}

Introduction

\section{Finite element method}

For the simple, one dimensional case, the exact solution to the
Helmholtz problem is know. Finding the exact solution for more complex
and higher dimensional problems turns out to be hard and impractical.
Often, an approximation to the exact solution suffices for engineering
purposes. A widly used method to find such approximations is know as
the \emph{Galerkin finite element method}.

Using this method, an approximation to the real solution is found by
transforming the problem in a system linear equations.  This results
in a sparse linear system for which many solving techniques have been
studied \cite{}. Although the linear systems resulting from this
method are sparse, for an accurate aproximation of solution to higher
dimensional or heavily oscillating problems a very large sytem can be
needed.

%When searching for an approximate solution to the Helmholtz problem
%using a Finite Element approach, we aim to approximate a function
%$u$ by constructing a finite linear system and solving the linear
%equations. For this approximation to be possible, we want to
%restrict our search space for $U(x) \approx u(x)$ to a finite dimensional
%function space.
%
%In this section, we will show how an approximation for $u$ can be found
%using the \emph{Galerkin finite element method} in both
%one and higher dimensions. We will give error estimates in both
%cases and we will address the problems that arise when
%dealing with heavily oscillating functions.

\subsection{One dimensional Helmholtz problem}

As an example, we will consider a simple one dimensional wave problem.
Suppose we have the following conditions for $u(x)$ on $[0, 1]$:
\begin{align}
    - \frac{d^2u}{dx^2} - k^2 u
      & = f, \quad \text{on}\ \Omega = (0, 1), \\
    u(0) & = 0, \label{eq:dirbc} \\
    \frac{du}{dx}(1) - i k u(1) & = 0. \label{eq:impbc} &
\end{align}
It can be shown that the exact solution to is given by
\begin{equation}
  u(x) = \frac{e^{ikx}}{k} \int^x_0 \sin(ks)f(s) ds
          + \frac{\sin(kx)}{k} \int^1_x e^{iks} f(s) ds,
\end{equation}
which is periodic with wavelenth $\lambda = \frac{2\pi}{k}$.

While this problem has a simple solution, exact solutions to more
complex and higher order problems are difficult, nigh impossible to
find in a simmilar matter. To find a good approximation, we have to
restrict the problem to one that can be solved with a computer.

\subsection{Galerking finite element method}

Since the original problem is infinite dimensional, we simplify the
problem by restricting the approximation of the real solution to a
finite dimensional search space. Suppose we have a bounded domain
$\Omega \subset \Reals^n$, $n = 1, 2, 3$.  We define the function
space $\LO$ of square integrable function on $\Omega$ by saying that
$f \in \LO$ if
\begin{equation}
  \lVert f \rVert := \left(
    \int_\Omega | f(x) |^2 d\Omega \right)^{\frac{1}{2}} < \infty.
\end{equation}
Furthermore, we say $f \in \HOzero$ if $f(0) = 0$ and
\begin{equation}
  \lVert \nabla f \rVert^2 + \lVert f \rVert^2 < \infty.
\end{equation}

For solving the above system with a finite element method, we will
first rewrite the problem in its weak form.  We multiply both sides
with a test function $v \in \HOzero$ and integrate afterwards to
obtain
\begin{equation}
  - \int^1_0 u''(x)v(x) - k^2u(x)v(x) dx = \int^1_0 f(x)v(x) dx.
\end{equation}
By substituting boundary condition \eqref{eq:impbc} and taking the fact
that $v(0) = (0)$ into consideration, we obtain
\begin{equation} \label{eq:weakprob}
  \int^1_0 u'(x)v'(x)dx - k^2 \int^1_0 u(x)v(x) dx - iku(1)v(1) = \int^1_0 f(x)v(x) dx.
\end{equation}
Instead of solving the above equation for the exact solution $u(x)$,
we want to approximate it by finding $U(x) \approx u(x)$. To do this,
we define a finite element mesh $X_h$ on $\Omega$, by
\begin{equation}
  X_h := \{ x_i ; x_i = ih, i = 0, 1, \ldots, N \},
\end{equation}
where $h = 1/N$.
We limit our search space to $S_h(0,1) \subset \HOzero$, the space of
piecewise continuous linear functions with nodal values at the points
in $X_h$, satisfying \eqref{eq:dirbc}. This function space is spanned
by the set of hat functions defined as
\begin{equation}
  \chi_j(x) = \begin{dcases*}
    \frac{1}{h} (x - x_{j - 1}), & $x \in [x_{j - 1}, x_j],$ \\
    \frac{1}{h} (x_{j + 1} - x), & $x \in [x_{j}, x_{j + 1}],$ \\
    0 & elsewhere,
  \end{dcases*}
\end{equation}
for $j = 1, 2, \ldots, N - 1$ and for $j = N$ by
\begin{equation}
  \chi_N(x) = \begin{dcases*}
    \frac{1}{h} (x - x_{j - 1}), & $x \in [x_{j - 1}, 1],$ \\
    0 & elsewhere.
  \end{dcases*}
\end{equation}
Now, if we require that both $U$ and $v$ are in  $S_h(0,1)$, then we
can write
\begin{equation}
  U(x) = \sum^N_{j = 1} u_j \chi_j(x),
\end{equation}
and \eqref{eq:weakprob} transforms to
\begin{equation}
  \sum^N_{j = 1} \left[ \int^1_0 \chi_j'(x) \chi_m'(x) dx
    - k^2 \int^1_0 \chi_j(x) \chi_m(x) dx \right] u_j
    - i k u_N \chi_m(1)
  =
    \int^1_0 f(x) \chi_m(x) dx,
\end{equation}
for $m = 1, 2, \ldots, N$.

\subsection{Linear system}

\subsection{Error estimates}


\section{Higher dimensional problem}

\subsection{Linear system}

\subsection{Mesh generation}


\section{Unbounded domains}

\subsection{Absorbing boundary conditions}


\section{Large wavenumbers and error estimates}

\end{document}
