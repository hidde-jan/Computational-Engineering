\documentclass[a4paper]{article}

\author{Hidde-Jan Jongsma}
\title{Finite Element Method}
\date{\today}

% Extra symbols and enviroments
\usepackage{amsmath}
\usepackage{amsfonts}
\usepackage{mathtools}

% Use full page
\usepackage{fullpage}

% Create graphics in LaTex
\usepackage{tikz}

% Extra commands
\newcommand{\Reals}{\mathbb{R}}

\newcommand{\dd}{\mathrm{d}}

\newcommand{\dun}{\frac{\partial u}{\partial n}}
\newcommand{\dux}{\frac{\partial u}{\partial x}}
\newcommand{\duxx}{\frac{\partial^2 u}{\partial x^2}}
\newcommand{\duxxx}{\frac{\partial^3 u}{\partial x^3}}

\newcommand{\duy}{\frac{\partial u}{\partial y}}
\newcommand{\duyy}{\frac{\partial^2 u}{\partial y^2}}
\newcommand{\duyyy}{\frac{\partial^3 u}{\partial y^3}}
\newcommand{\lapu}{\nabla^2 u}

\newcommand{\vct}{\mathbf}

\providecommand{\abs}[1]{\lvert#1\rvert}
\providecommand{\norm}[1]{\lVert#1\rVert}

\newcommand{\LO}{\ensuremath{L_2(\Omega)}}
\newcommand{\HO}{\ensuremath{H^1(\Omega)}}
\newcommand{\HOzero}{\ensuremath{H^1_{(0}(\Omega)}}


\begin{document}

\section{Higher dimensional problem}

For higher dimensional problems, again we consider the general
notations for the Helmholtz problem. For a domain $\Omega \in
\Reals^2$ with boundary $\Gamma$, the Helmholtz problem is stated as
follows
\begin{align}
  \Delta u + k^2 u & = 0, \quad \text{in}\ \Omega, \\
  \frac{\partial u}{\partial n} + \beta u & = g, \quad \text{on}\ \Gamma, \label{eq:2Dbc}
\end{align}
where $k \in \Reals$, $\beta \in \mathbb{C}$ and
${\partial}/{\partial n}$ is the outward normal derivative. Note
that
\begin{equation*}
  \frac{\partial u}{\partial n} = \nabla u \cdot n, \quad \label{eq:2DH}
  \Delta u = \lapu.
\end{equation*}
Again, we transform the problem to a weak formulation by first
multiplying \eqref{eq:2DH} by a test function $v \in \HO$ and
integrating it to get
\begin{equation} \label{eq:2Dint}
  \int_\Omega v \Delta u \ \dd\Omega + \int_\Omega k^2 u v \ \dd\Omega = 0.
\end{equation}
By applying Green's identity
\begin{equation*}
  \int_\Omega v \Delta u \ \dd \Omega
  + \int_\Omega \nabla v \cdot \nabla u \ \dd\Omega
  =
  \int_\Gamma v(\nabla u \cdot n) \ \dd\Gamma,
\end{equation*}
we find that
\begin{equation*}
  \int_\Gamma v \dun \ \dd\Gamma
  - \int_\Omega \nabla v \cdot \nabla u - k^2 u v \ \dd\Omega
  = 0.
\end{equation*}
From multiplying boundary equation \eqref{eq:2Dbc} with $v$ and
integrating, we obtain
\begin{equation*}
  \int_\Gamma v \dun \ \dd\Gamma
  =
  - \beta \int_\Gamma u v \ \dd\Gamma
  + \int_\Gamma v g \ \dd\Gamma.
\end{equation*}
By substituting these equations into \eqref{eq:2Dint} we get the weak
formulation of our problem; find $u \in \HO$ such that
\begin{equation}
  \int_\Omega \nabla v \cdot \nabla u - k^2 u v \ \dd\Omega
  + \beta \int_\Gamma uv \ \dd\Gamma
  = \int_\Gamma v g \ \dd\Gamma
\end{equation}
for all $v \in \HO$.

As for the one dimensional problem, we will restrict our search space
to a finite dimensional function space $V \subset \HO$. We define $V$
to be the span of basis functions $\chi_j$, $j = 1, 2, \ldots, N$.
When we require that $U, v \in V$, we are able to write
\begin{equation}
  U(x) = \sum^N_{j = 1} u_j \chi_j(x),
\end{equation}
where we have to determine the values of $u_j$. The problem can now be
reformulated as
\begin{equation}
  \sum^N_{j = 1} \left[
    \int_\Omega \nabla \chi_j \cdot \nabla \chi_m
    - k^2 \chi_j \chi_m \ \dd\Omega
    + \beta \int_\Gamma \chi_j \chi_m \ \dd\Gamma
  \right] u_j
  =
  \int_\Gamma \chi_m g \ \dd\Gamma,
\end{equation}
for $m = 1, 2, \ldots N$. Expressed as a linear system, this becomes
\begin{equation*}
  A \vct{u} = \vct{f},
\end{equation*}
where
\begin{equation*}
  A = \begin{pmatrix}
    a_{11} & a_{12} & \hdots  & a_{1N} \\
    a_{21} & a_{22} & \hdots  & a_{2N} \\
    \vdots & \vdots & \ddots  & \vdots \\
    a_{N1} & a_{N2} & \hdots  & a_{NN}
  \end{pmatrix},
\quad
\vct{u} = \begin{pmatrix}
    u_1 \\
    u_2 \\
    \vdots \\
    u_N
\end{pmatrix}
\quad \text{and} \quad
\vct{f} = \begin{pmatrix}
    \int_\Gamma \chi_1 g \ \dd\Gamma \\
    \int_\Gamma \chi_2 g \ \dd\Gamma \\
    \vdots \\
    \int_\Gamma \chi_N g \ \dd\Gamma \\
\end{pmatrix}.
\end{equation*}
Here, the matrix entries $a_{jm}$ of $A$ are defined as
\begin{equation*}
  a_{jm}
  = \int_\Omega \nabla \chi_j \cdot \nabla \chi_m
    - k^2 \chi_j \chi_m \ \dd\Omega
    + \beta \int_\Gamma \chi_j \chi_m \ \dd\Gamma.
\end{equation*}

\subsection{Linear system}

\subsection{Mesh generation}


\section{Unbounded domains}

\subsection{Absorbing boundary conditions}


\section{Large wavenumbers and error estimates}
\end{document}
