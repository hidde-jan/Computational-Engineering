\documentclass[a4paper]{article}

\author{Hidde-Jan Jongsma}
\title{Finite Element Method}
\date{\today}

\usepackage{amsmath}
\usepackage{amsfonts}
\usepackage{mathtools}

\usepackage{tikz}

\newcommand{\Reals}{\mathbb{R}}

\newcommand{\dux}{\frac{\partial u}{\partial x}}
\newcommand{\duxx}{\frac{\partial^2 u}{\partial x^2}}
\newcommand{\duxxx}{\frac{\partial^3 u}{\partial x^3}}

\newcommand{\duy}{\frac{\partial u}{\partial y}}
\newcommand{\duyy}{\frac{\partial^2 u}{\partial y^2}}
\newcommand{\duyyy}{\frac{\partial^3 u}{\partial y^3}}
\newcommand{\lapu}{\nabla^2 u}

\providecommand{\abs}[1]{\lvert#1\rvert}
\providecommand{\norm}[1]{\lVert#1\rVert}

\newcommand{\LO}{\ensuremath{L_2(\Omega)}}
\newcommand{\HO}{\ensuremath{H^1(\Omega)}}
\newcommand{\HOzero}{\ensuremath{H^1_{(0}(\Omega)}}


\begin{document}

Introduction

\section{Finite Element Method in one dimension}

When searching for an approximate solution to the Helmholtz problem
using a Finite Element approach, we aim to approximate a function
$u$ by constructing a finite linear system and solving the linear
equations. For this approximation to be possible, we want to
restrict our search space for $U(x) \approx u(x)$ to a finite dimensional
function space.

In this section, we will show how an approximation for $u$ can be found
using the \emph{Galerkin finite element method} in both
one and higher dimensions. We will give error estimates in both
cases and we will address the problems that arise when
dealing with heavily oscillating functions.

\subsection{Search space}

Suppose we have a bounded domain in $\Reals^n$, $n = 1, 2, 3$.
We define the function space $\LO$ of square integrable
function on $\Omega$ by saying that $f \in \LO$ if
\begin{equation}
  \lVert f \rVert := \left( \int_\Omega | f(x) |^2 d\Omega \right)^{\frac{1}{2}} < \infty.
\end{equation}
We define the \emph{Sobolev space} $\HO$ by saying $f \in \HO$ if
\begin{equation}
  \lVert \nabla f \rVert^2 + \lVert f \rVert^2 < \infty.
\end{equation}
Finaly, we say $f \in \HOzero$ if $f \in \HO$ and $f(0) = 0$.

\subsection{Exact solution}

As an example, we will consider a simple one dimensional wave
problem. Suppose we have the following conditions for
a time-harmonic function $u(x)$ on $[0, 1]$:
\begin{equation}
  \begin{dcases*}
    - \frac{d^2u}{dx^2} - k^2 u = f & on $\Omega = (0, 1)$ \\
    u(0) = 0 \label{eq:dirbc} & \\
    \frac{du}{dx}(1) - i k u(1) = 0 \label{eq:impbc} &
  \end{dcases*}
\end{equation}
where we have a Dirichlet boundary condition at $x = 0$ and
an impedance or Robin boundary equation in $x = 1$.
Here, the forcing term $f \in \LO$.

It can be shown that the exact solution to is given by
\begin{equation}
  u(x) = \frac{e^{ikx}}{k} \int^x_0 \sin(ks)f(s) ds
          + \frac{\sin(kx)}{k} \int^1_x e^{iks} f(s) ds,
\end{equation}
which is periodic with wavelenth $\lambda = \frac{2\pi}{k}$.
While this problem has a simple solution, exact solutions
to more complex and higher order problems are difficult,
nigh impossible to find in a simmilar matter.

For solving the above system with a finite element method,
we will first rewrite the problem in its weak form.
We multiply both sides with a test function $v \in \HOzero$
and integrate afterwards to obtain
\begin{equation}
  - \int^1_0 u''(x)v(x) - k^2u(x)v(x) dx = \int^1_0 f(x)v(x) dx.
\end{equation}
We integrate the first term by part to get
\begin{equation}
  [-u'(x)v(x)]^1_0 + \int^1_0 u'(x)v'(x)dx - k^2 \int^1_0 u(x)v(x) dx = \int^1_0 f(x)v(x) dx.
\end{equation}
By subsituting the impedance boundary condition \eqref{eq:impbc} and
taking the fact that $v(0) = (0)$ into consideration, we obtain
\begin{equation} \label{eq:weakprob}
  \int^1_0 u'(x)v'(x)dx - k^2 \int^1_0 u(x)v(x) dx - iku(1)v(1) = \int^1_0 f(x)v(x) dx.
\end{equation}

\subsection{Galerking finite element method}

To solve \eqref{eq:weakprob}, we will start by defining a finite element
mesh
\begin{equation}
  X_h := \{ x_i ; 0 = x_0 < x_1 < \hdots < x_N = 1 \},
\end{equation}
on $\Omega = (0, 1)$. For simplicity, we will use a \emph{uniform}
mesh, where all elements $x_i$ have size $h = 1/N$.
We limit our search space to $S_h(0,1) \subset \HOzero$ as the
space of piecewise continuous linear functions with nodal values
at points in $X_h$, satisfying \eqref{eq:dirbc} at $x = 0$.
This function space is spanned by the set of hat functions
defined as
\begin{equation}
  \chi_j(x) = \begin{dcases*}
    \frac{1}{h} (x - x_{j - 1}), & $x \in [x_{j - 1}, x_j],$ \\
    \frac{1}{h} (x_{j + 1} - x), & $x \in [x_{j}, x_{j + 1}],$ \\
    0 & elsewhere,
  \end{dcases*}
\end{equation}
for $j = 1, 2, \ldots, N - 1$ and for $j = N$ by
\begin{equation}
  \chi_N(x) = \begin{dcases*}
    \frac{1}{h} (x - x_{j - 1}), & $x \in [x_{j - 1}, 1],$ \\
    0 & elsewhere,
  \end{dcases*}
\end{equation}
Now we search for an approximate solution by requiring
$U, v \in S_h(0,1)$.

\subsection{Linear system}

\subsection{Error estimates}


\section{Higher dimensional problem}

\subsection{Linear system}

\subsection{Mesh generation}


\section{Unbounded domains}

\subsection{Absorbing boundary conditions}


\section{Large wavenumbers and error estimates}

\end{document}
